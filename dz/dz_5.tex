\documentclass[12pt,a4paper]{article}
\usepackage[croatian]{babel}
\usepackage[utf8]{inputenc}
\usepackage[top=20mm]{geometry}
\newcommand{\shell}[1]{\texttt{#1}}
\begin{document}
	\title{Domaća zadaća - 05}
	\author{Dino Lukman}
	\maketitle
	Za svaki zadatak treba napisati bash skriptu s rješenjem zadatka. Jedna točka zadatka se može rješiti s više redova i naredbi u skripti. \\

	Zadatak 1
   	\begin{itemize}
			\item Kreirajte datoteku imenovanu kao korisnik koji trenutačno koristi terminal. 
      \item U datoteku spremite osnovne podatke o korisniku (ime, home direktorij i ljuska) te podatak u kojim se grupama nalazi.
		\end{itemize}
	Zadatak 2
		\begin{itemize}
			\item Kreirajte korisnika \shell{okosl} s lozinkom \shell{okosl}.
			\item Kreirajte grupu \shell{linux} i dodajte novo kreiranog korisnika u grupu.
			\item Kreirajte direktorij \shell{/tmp/tmp\_etc} i u njega kopirajte sadržaj direktorija \shell{/etc}. Sljedeće tri točke će se odnositi na taj direktorij.
			\item Korisnicima u grupi \shell{linux} dozvolite modificiranje svih datoteka u direktoriju i poddirektorijima s ekstenzijom \shell{.conf}.
      \item Korisnicima u grupi \shell{linux} i svim ostalim korisnicima zabranite otvaranje poddirektorija.
      \item Kreirajte poddirektorij \shell{world\_writable} i dozvolite svima zapisivanje u direktorij. Dozvole postavite tako da svaka nova datoteka u tom poddirektoriju ima dozvolu pisanja za sve korisnike.
      \item Zabranite korisniku \shell{okosl} mogućnost prijave na sustav
		\end{itemize}
\end{document}
