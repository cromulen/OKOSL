\documentclass[12pt,a4paper]{article}
\usepackage[croatian]{babel}
\usepackage[utf8]{inputenc}
\usepackage[top=20mm]{geometry}
\newcommand{\shell}[1]{\texttt{#1}}
\begin{document}
  \title{Domaća zadaća - 03\vspace{-2em}}
  \maketitle
  Za svaki zadatak treba napisati bash skriptu s rješenjem zadatka. Jedna točka - jedna naredba - jedan redak. Skripta se mora izvršavati bez prekida osim u zadacima gdje su prekidi eksplicitno dozvoljeni.
  \subsection*{Zadatak 1}
  \begin{itemize}
    \item Stvorite datoteku koja sadrzi vas jmbag u prvom retku koristeci echo naredbu i preusmjeravanje. Datoteku nazovite personal_info.dat
    \item Pomocu cat naredbe i here document dodajte vase ime u redak ispod vaseg jmbaga.
  \end{itemize}
  \subsection*{Zadatak 2}
  \begin{itemize}
    \item Koliki je rezultat naredbe \shell{echo 12345 | wc -c}? Zasto?
  \end{itemize}
  \subsection*{Zadatak 3}
  \begin{itemize}
      Napomena: Za sve tocke ovog zadatka iskoristite datoteku: korisnici.dat. Datoteke koje nastaju kao proizvod 1. zadatka nije potrebno stavljati u git.
      Svaki redak u datoteci prvi.dat ima format <broj retka>:<ime>:<id korisnika>
    \item Sortirajte datoteku po broju retka i rezultat zapisite u korisnici.sortirano.dat
    \item Ispisite sva jedinstvena (imena koja se ponavljaju maksimalno jednom) u datoteku jedistveni_korisnici.dat
    \item Ispisite samo imena koja se ponavljaju (jedno ime po duplikatu je dovoljno) u datoteku nejedinstveni_korisnici.dat
  \end{itemize}
  \subsection*{Zadatak 4}
  \begin{itemize}
    \item Prebrojite sve rijeci koje sadrze znak \&  u datoteci /usr/share/dict
  \end{itemize}
  \subsection*{Zadatak 5}
  \begin{itemize}
    \item Locirajte datoteku jedinstveni_korisnici.dat (stvorenu u zadatku 3) koristeci naredbe find, locate i grep.   
  \end{itemize}
\end{document}
