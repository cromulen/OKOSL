\documentclass[12pt,a4paper]{article}
\usepackage[croatian]{babel}
\usepackage[utf8]{inputenc}
\usepackage[top=20mm]{geometry}
\usepackage{enumitem} 
\newcommand{\shell}[1]{\texttt{\textbf{#1}}}
\renewcommand*{\familydefault}{\sfdefault}
\renewcommand*{\sfdefault}{lmss}
\begin{document}
	\title{Laboratorijska vježba 3\\{\small Osnove korištenja operacijskog sustava Linux}\vspace{-2em}}
	\maketitle
	Za svaki zadatak potrebno je napisati jednu bash skriptu, a za lakše rješavanje zadataka možete korisititi bilo koju linux utility ili skirptu koju sami napišete. Kako biste lakše demonstrirali rješenja preporučamo da nakon svakog ključnog koraka u zadatku prikažete rezultate izvršavanja skripte. U tomu vam mogu pomoći sljedeće naredbe:
	\begin{description}[leftmargin=!,labelwidth=4em,itemsep=0em]
		\item[\shell{clear}] Čisti sadržaj terminala
		\item[\shell{read -p}] Čita podatak s tipkovnice; Zaustavlja izvođenje skripte
		\item[\shell{less}] Pager
	\end{description}
	
  \subsection*{Zadatak 1}
  Napisati skriptu koja prima 2 parametra preko standardnog ulaza. Prvi paramater je putanja do datoteke, a drugi je pozitivan broj (N).
  Skripta treba ispisati sadržaj datoteke N puta na standardni izlaz. Skriptu nazovite \shell{enlarge.sh}. Primjer pozivanja skripte je \shell{echo myfile.dat 3 | ./enlarge.sh}

	\subsection*{Zadatak 2}
  Za rjesavanje zadatka koristite datoteke \shell{lab3-input01.dat} i \shell{lab3-input02.dat}.
	\begin{itemize}
    \item U prvoj datoteci nalaze id i naziv (id:naziv), a drugoj id i veličina (id:veličina). Vaš zadatak je napisati skriptu \shell{merge.sh} koja će stvoriti treću datoteku sadržaja id:naziv:veličina.
	\end{itemize}
	
	\subsection*{Zadatak 3}
  Za rjesavanje zadatka koristite datoteke \shell{lab3-input03.dat i \shell{lab3-output03.dat}.
	\begin{itemize}
    \item Napisati skriptu koja ce na standardni ulaz primiti datoteku, te napraviti izmjene nad njome. Svaki redak datoteke jet tipa ključ:vrijednost. Datoteka je nesortirana, i parovi ključ:vrijednost javljaju se nekoliko puta. Zadatak je izmijeniti ulaz tako da se ključ pojavljuje samo jednom, a sve vrijednosti za taj ključ su u istom redu, odvojene zarezom. Primjer ulaza i izlaza su datoteke input03 i output03. 
	\end{itemize}
	
	\subsection*{Zadatak 4}
	\begin{itemize}
		\item Stvoriti direktorij /home/studenti, kreirati grupu studenti i omoguciti joj ekluzivna prava pisanja i čitanja nad direktorijem.
    \item Napisati skriptu koja će stvarati korisnike na vašem sustavu i pritom im stvarati \shell{home} unutar studenti direktorija, kao i dodati ih u grupu studenti. 
    \item Osim dodavanja u grupu potrebno je urediti i template za \shell{home} direktorij (skel folder). Direktoriji koji svaki student treba imati su: Documents, Github i Shared. Direktorij Shared neka bude simbolički link na direktorij \shell{/home/public}. Dozvole nad public direktorijem nek budu iste kao i od \shell{/tmp} direktorija.
	\end{itemize}
  
	\subsection*{Zadatak 5}
  Simulirana je obrada velike datoteke vlastitom skriptom. (Npr. cat big\_file.dat | ./slow\_script.sh).
	\begin{itemize}
    \item Nakon pokretanja procesa, postavljen je  process u pozadinu (background). Kako ste to napravili?
    \item Poslali ste signal SIGSTOP procesu. Kako ste to napravili? Koji ste process zaustavili (čitanje ili obradu)? Što je bolje?
    \item Nastavili ste izvodenje procesa. Kako ste to napravili?
    \item Ugasili ste terminal u kojem je pokrenuta skripta (u pozadini). Što se dogodi s procesom? Zašto?
    \item Pokrenuli ste process (prednji plan) i ugasili terminal. Skripta je nastavila izvođenje. Što ste napravili/koristili? (Dovoljna je navesti jednu stvar)
	\end{itemize}

\subsection*{Zadatak 6}
  Koristeci Here documents stvoriti datoteku sadrzaja:
  \begin{verbatim}
  <!DOCTYPE html>
  <html>
    <head>
      <title>Hi there</title>
    </head>
    <body>
      This is a page
      a simple page
    </body>
  </html>
  \end{verbatim}
  \begin{itemize}
    \item Ispisati sav sadržaj izmedu HTML oznake \textbf{html}.
    \item Ispisati sav sadržaj datoteke, bez HTML oznaka.
  \end{itemize}

\end{document}
