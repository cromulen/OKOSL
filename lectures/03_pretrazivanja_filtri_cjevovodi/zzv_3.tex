\documentclass{beamer}

\usepackage[english]{babel}
\usepackage[utf8]{inputenc}
\usepackage{listings}
\usepackage{datetime}
\usepackage{graphics}
\usepackage{fancybox}
\usepackage{color}
\usepackage[normalem]{ulem}
\usepackage{hyperref}
\usetheme{CambridgeUS}
\usecolortheme{seagull}
% Changing of bullet foreground color not possible if {itemize item}[ball]
\DefineNamedColor{named}{Purple}{cmyk}{0.52,0.97,0,0.55}
\setbeamertemplate{itemize item}[triangle]
\setbeamercolor{title}{fg=Purple}
\setbeamercolor{frametitle}{fg=Purple}
\setbeamercolor{itemize item}{fg=Purple}
\setbeamercolor{section number projected}{bg=Purple,fg=white}
\setbeamercolor{subsection number projected}{bg=Purple}

\renewcommand{\dateseparator}{.}
\newcommand{\todayiso}{\twodigit\day \dateseparator \twodigit\month \dateseparator \the\year}
\newcommand{\shell}[1]{\texttt{#1}}
\title{Osnove korištenja operacijskog sustava Linux}
\subtitle{02. Zadaci za vježbu}
\author[Sabrina Miškulin, Gregor Orlić, Marin Petričević, Leonard Volarić Horvat]{Sabrina Miškulin, Gregor Orlić, Marin Petričević, Leonard Volarić Horvat\\{\small Nositelj: dr. sc. Stjepan Groš}}
\institute[FER]{Sveučilište u Zagrebu \\
				Fakultet elektrotehnike i računarstva}

\date{\todayiso}

\begin{document}
    %\beamerdefaultoverlayspecification{<+->}
{
\setbeamertemplate{headline}[] % still there but empty
\setbeamertemplate{footline}{}

\begin{frame}
\maketitle
\end{frame}
}

\begin{frame}[t]
\frametitle{Zadatak 1}
\begin{itemize}
\item Prvi zadatak
\begin{itemize}
	\item Kreirati datoteku "Zadatak1" u /tmp direktoriju. U Zadatak1 upisati soritiran /etc/passwd po abecedi.
	\item Kreirati datoteku "Zadatak2" u /tmp direktoriju. U Zadatak2 upisati sve linije iz Zadatak1 koje sadrže riječ "false".
	\item Nadovezati Zadatak2 na Zadatak1 i upisati u /tmp/Zadatak3.
	\item U /home/Zadatak4 upisati sve jedinstvene redove iz Zadatak3.
	\item Pomoću naredbe find pronaći Zadatak4.
	\item Pomoću naredbe locate pronaći Zadatak4.
    \item Ispisati Zadatak4 na stderr.
    \item Izbrisati koristene datoteke.
\end{itemize}
\end{itemize}
\end{frame}

\begin{frame}[t]
\frametitle{Zadatak 2}
\begin{itemize}
\item Drugi zadatak
\begin{itemize}
	\item Sve se obavlja u home direktoriju!
	\item Kreirati praznu datoteku "Napuni". U datoteku "Napuni" treba prebaciti zadnjih
500 riječi iz /usr/share/dict/words.
	\item Iz "Napuni" ispisati samo one riječi koje pocinju sa zoo.
	\item Ispisati ukupni broj riječi i slova u "Napuni".
	\item Preimenovat "Napuni" u "Isprazni". "Isprazni" treba isprazniti u novi "Napuni2" koji je sortiran silazno [z,y,x,w,v...].
	\item Od "Napuni2" ispisati samo prva tri slova svake riječi, s time da treba ispisati one kombinacije slova koje se ne ponavljaju vise od jednom.
	\item Izbrisati "Napuni2" i "Isprazni".

\end{itemize}
\end{itemize}
\end{frame}

\begin{frame}[t]
\frametitle{Zadatak 3}
\begin{itemize}
\item Treci zadatak
\begin{itemize}
	\item Pomocu find naredbe ispisati sve postojece direktorije u datoteku "znamOvo", a sve greske u "neZnamOvo".
	\item Prebaciti prvih i zadnjih 20 linija iz "znamOvo" na kraj datoteke "neZnamOvo".


\end{itemize}
\end{itemize}

\end{frame}


\end{document}
