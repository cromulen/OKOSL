% mozda grupirati slajdove za pregled sadrzaja datoteke (ima 1 ispred 
% ostalih)

\documentclass{beamer}

\usepackage[english]{babel}
\usepackage[utf8]{inputenc}
\usepackage{listings}
\usepackage{datetime}
\usepackage{graphics}
\usepackage{fancybox}
\usepackage{color}
\usepackage[normalem]{ulem}
\usepackage{tikz}
\usepackage{courier}
\usetikzlibrary{shapes,arrows}
\usetheme{CambridgeUS}
\usecolortheme{seagull}
% Changing of bullet foreground color not possible if {itemize item}[ball]
\DefineNamedColor{named}{Purple}{cmyk}{0.52,0.97,0,0.55}
\setbeamertemplate{itemize item}[triangle]
\setbeamercolor{title}{fg=Purple}
\setbeamercolor{frametitle}{fg=Purple}
\setbeamercolor{itemize item}{fg=Purple}
\setbeamercolor{section number projected}{bg=Purple,fg=white}
\setbeamercolor{subsection number projected}{bg=Purple}

\renewcommand{\dateseparator}{.}
\newcommand{\todayiso}{\twodigit\day \dateseparator \twodigit\month \dateseparator \the\year}
\newcommand{\shell}[1]{\texttt{#1}}

\title{Osnove korištenja operacijskog sustava Linux}
\subtitle{04.Pogled unutar datoteke} 
\author[Goran Cetušić]{Goran Cetušić\\{\small Nositelj: dr. sc. Stjepan Groš}}
\institute[FER]{Sveučilište u Zagrebu \\
				Fakultet elektrotehnike i računarstva}
				
\date{\todayiso}

\begin{document}
    %\beamerdefaultoverlayspecification{<+->}
{
\setbeamertemplate{headline}[] % still there but empty
\setbeamertemplate{footline}{}

\begin{frame}
\maketitle
\end{frame}
}

\begin{frame}
\frametitle{Sadržaj}
\tableofcontents
\end{frame}

\section{Unix datoteke}
\begin{frame}[t]
\frametitle{Unix datoteke (1)}
\begin{itemize}
  \item Na Unix sustavima je sve datoteka
  \item Posebne vrste datoteka postoje zbog različitih pristupa dijelova
        sustava
  \begin{itemize}
    \item Tipkovnice
    \item Diskovi
    \item Zvučne kartice
    \item \ldots
  \end{itemize}
\end{itemize}
\end{frame} 

\begin{frame}[t]
\frametitle{Unix datoteke (2)}
\begin{itemize}
  \item Vrsta datoteke određuje se naredbom \shell{ls}
  \item Prvi znak pri dugom ispisu određuje tip datoteke
  \item Primjer
  \begin{itemize}
    \item Stvoriti direktorij i ispisati informacije o njemu
  \end{itemize}
  \item[] \shell{\$ mkdir direktorij}
  \item[] \shell{\$ ls -ld direktorij}
  \item[] \shell{\footnotesize\textbf{d}rwxr-xr-x 2 cetko cetko 4096 
                          2010-10-28 12:39 direktorij/}
\end{itemize}
\end{frame}

\begin{frame}[t]
\frametitle{Obična datoteka} 
\begin{itemize}
  \item Standardna datoteka, prva asocijacija na riječ datoteka
  \item Prazna datoteka stvara se naredbom \shell{touch} bez argumenata
  \item Zadatak
  \begin{itemize}
    \item Stvoriti datoteku \textbf{a}
    \item Izlistati datoteku
  \end{itemize}
\end{itemize}
\end{frame}


\begin{frame}[t]
\frametitle{Imenovani cjevovod}
\begin{itemize}
  \item engl. \emph{named pipe}
  \item Cjevovodi služe za povezivanje izlaza i ulaza dva procesa (engl. 
        \emph{interprocess communication})
  \begin{itemize}
    \item Obični cjevovodi postoje samo dok traje komunikacija
    \item Imenovani cjevovodi postoje kao datoteke na sustavu za
          komunikaciju procesa po potrebi
  \end{itemize}
  \item Označava se znakom \textbf{p}
  \item Više o cjevovodima kasnije
\end{itemize}
\end{frame}

\begin{frame}[t]
\frametitle{Direktorij}
\begin{itemize}
  \item Uz svaki direktorij vezana je lista s pripadajućim datotekama
  \item Okosnica strukture sustava
  \item Direktorij obilježava znak \textbf{d} kod ispisa naredbom
        \shell{ls -ld} 
\end{itemize}
\end{frame}

\begin{frame}[t]
\frametitle{Priključnica}
\begin{itemize}
  \item engl. \emph{socket}
  \item Služi za komununikaciju procesa preko mreže 
  \begin{itemize}
    \item Privremene datoteke
  \end{itemize}
  \item Označava se znakom \textbf{s}
  \item O priključnicama neće biti daljnjih tema
\end{itemize}
\end{frame}

\section{Datoteke uređaja}
\begin{frame}[t]
\frametitle{Datoteke uređaja (1)}
\begin{itemize}
  \item Datoteke uređaja (engl. \emph{device files}) služe za komunikaciju
        s vanjskim uređajima 
  \begin{itemize}
    \item Nalaze se u \shell{/dev} direktoriju
  \end{itemize}
  \item Uređaji komuniciraju sa sustavom u blokovima ili znak po znak
  \item Zadatak
  \begin{itemize}
    \item Izlistati direktorij \shell{/dev} 
  \end{itemize}
\end{itemize}
\end{frame}

\begin{frame}[t]
\frametitle{Datoteke uređaja (2)}
\begin{itemize}
  \item Blok datoteke
  \begin{itemize}
    \item Prijenos podataka odvija se u blokovima 
    \item Diskovi, CD-ROM uređaji i memorije
  \end{itemize}
  \item Označavaju se znakom \textbf{b}
  \item Znakovne (engl. \emph{char}) datoteke
  \begin{itemize}
    \item Prijenos podataka znak po znak
    \item Tipkovnice i zvučne kartice
  \end{itemize}
  \item Označavaju se znakom \textbf{c}
\end{itemize}
\end{frame}

\section{Simboličke poveznice}
\begin{frame}[t]
\frametitle{Simboličke poveznice (1)}
\begin{itemize}
  \item engl. \emph{symbolic link}
  \item Poveznice služe za brže pristupanje podacima
  \begin{itemize}
    \item Poput prečaca (engl. \emph{shortcut}) na Windows sustavu
  \end{itemize}
  \item Naredbom \shell{ln -s} stvaraju se simboličke poveznice
  \item Zadatak
  \begin{itemize}
    \item U matičnom direktoriju stvoriti simboličku poveznicu \textbf{b}
          na datoteku \textbf{a} i izlistati poveznicu
  \end{itemize}
\end{itemize}
\end{frame}

\begin{frame}[t]
\frametitle{Simboličke poveznice (2)}
\begin{itemize}
  \item Rješenje glasi otprilike ovako
  \begin{itemize}
    \item[] \shell{\$ ln -s a b}
    \item[] \shell{\$ ls -l b}
    \item[] \shell{\footnotesize lrwxrwxrwx 1 cetko cetko 1 2010-10-28 19:18 b -> a}
  \end{itemize}
  \item Poveznice mogu referencirati apsolutne i relativne putanje 
  \item Brisanjem simboličke poveznice podaci ostaju na sustavu
  \item Zadatak: Obrisati poveznicu \textbf{b}
\end{itemize}
\end{frame}


\begin{frame}[t]
\frametitle{Pregled sadržaja datoteke}
\begin{itemize}
  \item Ako želimo vidjeti sadržaj neke datoteke na zaslonu možemo
        upotrijebiti naredbu \shell{cat}
  \begin{itemize}
    \item Sintaksa naredbe je 
    \item[] \shell{cat \textless ime datoteke\textgreater}
  \end{itemize}
  \item Zadatak
  \begin{itemize}
    \item Pogledati sadržaj sljedećih datoteka
    \item[] \shell{/etc/shells}
    \item[] \shell{/etc/terminfo}
    \item[] \shell{/bin/bash}
  \end{itemize}
\end{itemize}
\end{frame}

\section{MAC vremena}
\begin{frame}[t]
\frametitle{MAC vremena (1)}
\begin{itemize}
  \item Svaka datoteka/direktorij ima definirana tri vremena
  \begin{itemize}
    \item Vrijeme zadnje promjene (\emph{mtime})
    \begin{itemize}
      \item Naredba \shell{ls} ispisuje ovo vrijeme ako se drugačije 
               ne kaže!
    \end{itemize}
    \item Vrijeme zadnjeg pristupa (\emph{atime})
    \begin{itemize} 
      \item Naredba \shell{ls} će ispisati ovo vrijeme opcijom
            \shell{-u}
    \end{itemize}
    \item Vrijeme zadnje promjene metainformacije (\emph{ctime})
    \begin{itemize}
      \item Ovo vrijeme se ispisuje opcijom \shell{-c}
    \end{itemize}
  \end{itemize}
\end{itemize}
\end{frame}

\begin{frame}[t]
\frametitle{MAC vremena (2)}
\begin{itemize}
  \item Detaljne informacije o datotekama ispisuje naredba \shell{stat},
        uključujući i MAC vremena
  \item Zadatak
  \begin{itemize}
    \item Prvo naredbom \shell{ls}, a zatim \shell{stat} provjeriti MAC
          vremena
  \end{itemize}
\end{itemize}
\end{frame}

\begin{frame}[t]
\frametitle{MAC vremena (3)}
\begin{itemize}
  \item Naredba \shell{touch} sva vremena postavlja na trenutno
  \item Zadatak (koristiti \shell{stat}, \shell{touch} i \shell{cat}) 
  \begin{itemize}
    \item Izlistati vremena datoteke \textbf{a}
    \item Izlistati sadržaj datoteke \textbf{a}
    \item Izlistati vremena datoteke \textbf{a}
    \item Ažurirati vremena datoteke \textbf{a}
    \item Izlistati vremena datoteke \textbf{a}
  \end{itemize}
\end{itemize}
\end{frame}

\begin{frame}[t]
\frametitle{Naredba \shell{file} (1)}
\begin{itemize}
  \item Ponekad imamo na raspolaganju datoteku za koju ne znamo kakvog je
        tipa
  \item U tom slučaju na raspolaganju nam je naredba \shell{file}
  \begin{itemize}
    \item Ona pokušava odrediti vrstu datoteke na osnovu baze tipova 
          datoteka
  \end{itemize}
  \item Sintaksa je:
  \begin{itemize}
    \item[] \shell{file <ime datoteke>}
    \end{itemize}
\end{itemize}
\end{frame}  

\begin{frame}[t]
\frametitle{Naredba \shell{file} (2)}
\begin{itemize}
  \item Prilično kompleksna naredba koja ne daje točne rezultate u 100\% 
        slučajeva, ali je ipak izuzetno korisna 
  \item Napomena: tip datoteke u ovom slučaju se razlikuje od definicije
        Unix datoteka
  \begin{itemize}
    \item Pomoću tipa datoteke određuje se koja aplikacija je zadužena za
          pristupanje datoteci
  \end{itemize}
  \item Unix/Linux ne prepoznaje tipove po ekstenzijama
  \begin{itemize}
    \item Svaka ``ekstenzija'' je samo dio imena datoteke
  \end{itemize}
\end{itemize}
\end{frame}

\begin{frame}[t]
\frametitle{Naredba \shell{file} (3)}
\begin{itemize}
  \item Zadatak: Provjeriti tip sljedećih datoteka (i direktorija)
  \begin{itemize}
    \item[] \shell{/bin/sh}
    \item[] \shell{/bin/bash}
    \item[] \shell{/etc/passwd}
    \item[] \shell{/usr/bin}
    \item[] \shell{/usr/include/stdio.h}
  \end{itemize}
  \item Skinuti s Interneta neku datoteku i provjeriti njen tip!
\end{itemize}
\end{frame}
    
\section{Pregled sadržaja datoteke}
\begin{frame}[t]
\frametitle{Pregled dijela sadržaja datoteke (1)}
\begin{itemize}
  \item Ako je datoteka prevelika nećemo ništa vidjeti
  \begin{itemize}
    \item Primjer datoteke \shell{/usr/include/stdio.h}
    \item Ako je datoteka prevelika, a zanima nas samo prvih N redaka, 
          možemo upotrijebiti naredbu \shell{head}
  \end{itemize}
  \item Sintaksa datoteke je:
  \begin{itemize}
    \item[] \shell{head [-n <n>] <ime datoteke>}
    \item Ako se ne navede opcija, ispisuje prvih 10 redaka, inače ispisuje          prvih N redaka
  \end{itemize}
  \item Ako je \shell{<n>} negativan, ispisuje sve osim zadnjih n redaka
\end{itemize}
\end{frame}
  
\begin{frame}[t]
\frametitle{Pregled dijela sadržaja datoteke (2)}
\begin{itemize}
  \item Zadatak
  \begin{itemize}
    \item Pogledati prvih 10 redaka datoteke \shell{/etc/passwd}
    \item Pogledati prvih 5 redaka datoteke \shell{/etc/apt/sources.list}
  \end{itemize}
\end{itemize}
\end{frame}

\begin{frame}[t]
\frametitle{Pregled dijela sadržaja datoteke (3)}
\begin{itemize}
  \item Ako nas zanima nas samo zadnjih N redaka,možemo upotrijebiti 
        naredbu \shell{tail}
  \begin{itemize}
    \item Sintaksa datoteke je:
    \item[] \shell{tail [-n <n>] <ime datoteke>}
  \end{itemize}
  \item Ako se ne navede opcija, ispisuje zadnjih 10 redaka, inače ispisuje
        zadnjih N redaka
  \item Ako je \shell{<n>} pozitivan (ima znak +) ispisuje od zadanog 
        retka do kraja!
\end{itemize}
\end{frame}

\begin{frame}[t]
\frametitle{Pregled dijela sadržaja datoteke (4)}
\begin{itemize}
  \item Zadatak
  \begin{itemize}
    \item Pogledati zadnjih 10 redaka datoteke \shell{/etc/passwd}
    \item Pogledati zadnjih 20 redaka datoteke
          \shell{/etc/apt/sources.list}
  \end{itemize}
\end{itemize}
\end{frame}

\begin{frame}[t]
\frametitle{Pregledavanje datoteka po stranicama (1)}
\begin{itemize}
  \item Ako želimo pregledavati sadržaj datoteke stranicu po stranicu, na
        raspolaganju imamo naredbe \shell{less} ili \shell{more}
  \item ``Stranica'' je količina teksta koja stane na jedan ekran!
  \begin{itemize}
    \item \shell{less} je novija varijanta sa većim mogućnostima
  \end{itemize}
  \item Standard na Linuxu, ali nije dostupna na komercijalnim Unix 
        sustavima
  \item Obje naredbe kao argument primaju ime datoteke
  \begin{itemize}
    \item Izlazak iz pregledavanja je s tipkom q
    \item Naredba \shell{man} koristi te programe za prikaz uputa!
  \end{itemize}
\end{itemize}
\end{frame}

\begin{frame}[t]
\frametitle{Pregledavanje datoteka po stranicama (2)}
\begin{itemize}
  \item Zadatak
  \begin{itemize}
    \item Naredbom \shell{less} pogledati sadržaje sljedećih datoteka
    \item[] \shell{/etc/bash\_completion}
    \item[] \shell{/etc/init.d/networking}
  \end{itemize}
\end{itemize}
\end{frame}

\section{Mijenjanje sadržaja datoteka}
\begin{frame}[t]
\frametitle{Mijenjanje sadržaja datoteka}
\begin{itemize}
  \item Postoji nekoliko tekstualnih editora na Linuxu 
  \item Često korišteni su \shell{nano} i \shell{vim}
  \item Editor \shell{nano} je jednostavan za uporebu i najrašireniji 
        na Linuxu
  \item Zadatak
  \begin{itemize}
    \item Naredbom \shell{nano} upisati “Osnove Linuxa” u datoteku 
          \textbf{a}
    \item Izlistati MAC vremena dotične datoteke
  \end{itemize}
\end{itemize}
\end{frame}

\begin{frame}[t]
\frametitle{Naredba \shell{vim} (1)}
\begin{itemize}
  \item \shell{vim} je moćan editor tekstualnih datoteka
  \begin{itemize}
    \item Naprednija verzija \shell{vi} editora
    \item Osnovne funkcionalnosti su jednake
  \end{itemize}
  \item Može biti neintuitivan za početnike
  \item Podržava tri načina rada (engl. \emph{mode})
  \begin{itemize}
    \item \emph{command mode}
    \item \emph{insert mode}
    \item \emph{last line mode}
  \end{itemize}
\end{itemize}
\end{frame}

\begin{frame}[t]
\frametitle{Naredba \shell{vim} (2)}
\begin{itemize}
  \item command mode
  \begin{itemize}
    \item x -- brisanje znaka ispod kursora
    \item dd -- brisanje linije
    \item yy -- kopiranje (engl. \emph{yank}) linije
    \item p --  ubacivanje kopiranog
    \item u -- poništavanje prethodne akcije
    \item /\textless ključna riječ\textgreater -- pretraživanje
    \item n,p -- sljedeće i prethodno pojavljivanje riječi
  \end{itemize}
\end{itemize}
\end{frame}

\begin{frame}[t]
\frametitle{Naredba \shell{vim} (3)}
\begin{itemize}
  \item insert mode
  \begin{itemize}
    \item a -- ubacivanje teksta prije kursora
    \item i -- ubacivanje teksta nakon kursora
  \end{itemize}
  \item Vraćamo se u komandni način rada sa ESC
  \item Zadatak
  \begin{itemize}
    \item Naredbom \shell{vim} obrisati prvu liniju datoteke a
    \item Upisati ``Korištenje vima 101''
  \end{itemize}
\end{itemize}
\end{frame}

\begin{frame}[t]
\frametitle{Naredba \shell{vim} (4)}
\begin{itemize}
  \item last line mode
  \item Iz komandnog načina rada se prebacuje upisivanjem dvotočke :
  \begin{itemize}
    \item[] :q -- izlazak iz editora
    \item[] :q! -- zanemarivanje promjena i izlazak iz editora
    \item[] :wq -- zapisivanje promjena i izlazak iz editora
    \item \shell{vim} neće dopustiti samo izlazak iz editora ako su rađene 
          promjene -- moraju biti zanemarene ili zapisane
  \end{itemize}
\end{itemize}
\end{frame}

\begin{frame}[t]
\frametitle{Naredba \shell{vim} (5)}
\begin{itemize}
  \item Zadatak
  \begin{itemize}
    \item Spremiti promijenjenu datoteku \textbf{a}
    \item Ponovno otvoriti, promijeniti i pokušati izaći bez spremanja 
          promjena
  \end{itemize}
  \item vimtutor – program za učenje kako koristiti \shell{vim}
\end{itemize}
\end{frame}

\begin{frame}[t]
\frametitle{Literatura}
\begin{itemize}
  \item \url{http://www.tuxfiles.org/linuxhelp/vimcheat.html}
  \item \url{http://www.unixtutorial.org/2007/09/unix-file-types/}
  \item \url{http://en.wikipedia.org/wiki/MAC\_times}
  \item man file
  \item \url{http://kb.iu.edu/data/abbe.html}
\end{itemize}
\end{frame}

\end{document}
