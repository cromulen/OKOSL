\documentclass[12pt,a4paper]{article}
\usepackage[croatian]{babel}
\usepackage[utf8]{inputenc}
\usepackage[top=20mm]{geometry}
\newcommand{\shell}[1]{\texttt{#1}}
\begin{document}
	\title{Domaća zadaća - 02}
	\author{Dominik Barbarić}
	\maketitle
	Rješenje zadatka je potrebno upisati u \shell{.sh} datoteku. Jedna točka - jedna naredba - jedan redak.
	\begin{itemize}
	\item[]
		\item Stvorite simboličku poveznicu na datoteku \shell{/etc/passwd} i nazovite je \shell{korisnici}.
		\item Odredite tip datoteke \shell{korisnici} korištenjem naredbe \shell{file}.
		\item Naredbom \shell{wc} doznajte ukupan broj korisnika na sistemu.
		\item Ispišite sve osim zadnjih 5 redova datoteke \shell{korisnici}.
		\item Koristeći naredbu \shell{touch} datoteci \shell{korisnici} promijenite \emph{samo} modification time.
		\item Naredbom \shell{stat} ispišite informacije o datoteci \shell{korisnici}.
    \item Ispišite sadržaj /var/log/syslog datoteke \emph{uz pracenje} novih promjene u njoj.
		\item Naredbom \shell{df} provjerite koji je blok uređaj montiran na lokaciju \shell{/}.
		\item Provjerite veličinu blok datoteke montirane na lokaciju \shell{/}.
		\item Saznajte ukupno diskovno zauzeće vašeg matičnog direktorija.
    \item Ispisite datoteke vece od 2MB iz vašeg matičnog direktorija. 
	\end{itemize}
  Za one koji žele znati više: Napišite bash skriptu čita nešto sa standardnog ulaza i ispisuje na standardni izlaz. Nad pročitanim nemora raditi izmjene.
\end{document}
