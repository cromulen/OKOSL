\documentclass[12pt,a4paper]{article}
\usepackage[croatian]{babel}
\usepackage[utf8]{inputenc}
\usepackage[top=20mm]{geometry}
\newcommand{\shell}[1]{\texttt{#1}}
\begin{document}
	\title{Domaća zadaća - 02\vspace{-2em}}
	\maketitle
	Za svaki zadatak treba napisati bash skriptu s rješenjem zadatka. Jedna točka - jedna naredba - jedan redak. Skripta se mora izvršavati bez prekida osim u zadacima gdje su prekidi eksplicitno dozvoljeni.
	\subsection*{Zadatak 1}
	\begin{itemize}
		\item Stvorite simboličku poveznicu na datoteku \shell{/etc/passwd} i nazovite je \shell{korisnici}.
		\item[] /etc/passwd je datoteka koja sadrži popis svih korisnika na računalu u formatu jedan redak - jedan korisnik.
		\item Odredite tip datoteke \shell{korisnici} korištenjem naredbe \shell{file}.
		\item Prethodnu naredbu ponovite korištenjem opcije -L
	\end{itemize}
	U komentaru skripte objasnite rezultate prethonih dviju naredbi. (Pomoć: man stranice)
	\subsection*{Zadatak 2}
	\begin{itemize}
		\item Naredbom \shell{wc} doznajte ukupan broj korisnika na sistemu.
		\item Ispišite sadržaj datoteke \shell{korisnici} bez zadnjih 5 redova.
		\item Koristeći naredbu \shell{touch} datoteci \shell{korisnici} promijenite \emph{samo} modification time.
		\item Pokrenite naredbu koja ispisuje broj hard linkova za datoteku /etc/passwd.
	\end{itemize}
	U komentar odgovorite mijenja li se taj broj kreiranjem symbolic linkova kao u prvoj točki zadaće.
	\subsection*{Zadatak 3}
	\begin{itemize}
		\item Ispišite sadržaj /var/log/syslog datoteke \emph{uz stalno praćenje} novih promjena u datoteci. Naredba smije zaustaviti izvršavanje cijele skripte sa zadaćom. U komentar napišite kako prekinuti ispis novih podataka.
	\end{itemize}
	\subsection*{Zadatak 4}
	\begin{itemize}
		\item Naredbom \shell{df} provjerite koji je blok uređaj montiran na lokaciju \shell{/}.
		\item Provjerite veličinu blok datoteke montirane na lokaciju \shell{/}.
		\item Saznajte ukupno diskovno zauzeće vašeg matičnog direktorija.
		\item Ispišite datoteke vece od 2MB iz vašeg matičnog direktorija. 
	\end{itemize}
\end{document}
