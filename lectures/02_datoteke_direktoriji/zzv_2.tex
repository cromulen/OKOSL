\documentclass{beamer}

\usepackage[english]{babel}
\usepackage[utf8]{inputenc}
\usepackage{listings}
\usepackage{datetime}
\usepackage{graphics}
\usepackage{fancybox}
\usepackage{color}
\usepackage[normalem]{ulem}
\usepackage{hyperref}
\usetheme{CambridgeUS}
\usecolortheme{seagull}
% Changing of bullet foreground color not possible if {itemize item}[ball]
\DefineNamedColor{named}{Purple}{cmyk}{0.52,0.97,0,0.55}
\setbeamertemplate{itemize item}[triangle]
\setbeamercolor{title}{fg=Purple}
\setbeamercolor{frametitle}{fg=Purple}
\setbeamercolor{itemize item}{fg=Purple}
\setbeamercolor{section number projected}{bg=Purple,fg=white}
\setbeamercolor{subsection number projected}{bg=Purple}

\renewcommand{\dateseparator}{.}
\newcommand{\todayiso}{\twodigit\day \dateseparator \twodigit\month \dateseparator \the\year}
\newcommand{\shell}[1]{\texttt{#1}}
\title{Osnove korištenja operacijskog sustava Linux}
\subtitle{02. Zadaci za vježbu}
\author[Antun Aleksa, Josip Žuljević]{Antun Aleksa, Josip Žuljević\\{\small Nositelj: dr. sc. Stjepan Groš}}
\institute[FER]{Sveučilište u Zagrebu \\
				Fakultet elektrotehnike i računarstva}
				
\date{\todayiso}

\begin{document}
    %\beamerdefaultoverlayspecification{<+->}
{
\setbeamertemplate{headline}[] % still there but empty
\setbeamertemplate{footline}{}

\begin{frame}
\maketitle
\end{frame}
}

\begin{frame}[t]
\frametitle{Zadatak 1}
\begin{itemize}
  \item Prvi zadatak:
  \begin{itemize}
		\item Napraviti direktorij \shell{Paul}
		\item U njemu napravite praznu datoteku \shell{Ringo}
		\item Kopirajte datoteku \shell{Ringo} na desktop
		\item Ispišite zadnje vrijeme promjene direktorija \shell{Paul}
		\item Izbrišite datoteku \shell{Ringo} u direktoriju \shell{Paul}
  \end{itemize}
\end{itemize}
\end{frame}

\begin{frame}[t]
\frametitle{Zadatak 2}
\begin{itemize}
	\item Drugi zadatak:
	\begin{itemize}
		\item Napravite prazne datoteke \shell{John} i \shell{George}
		\item U \shell{John} upišite 10 redova teksta, a u \shell{George} 5 redova
		\item Obadvije datoteke premjestite u novi direktorij \shell{One}
		\item Ispišite prva 3 reda datoteke \shell{George}
		\item Koristeći opciju za negativan broj redaka ispišite prvih 7 redova datoteke \shell{John}
		\item Izbrišite sve stvorene datoteke i direktorije
	\end{itemize}
\end{itemize}
\end{frame}


\begin{frame}[t]
\frametitle{Zadatak 3}
\begin{itemize}
	\item Treći zadatak:
	\begin{itemize}
		\item Napravite direktorij \shell{Jules}
		\item U njemu stvorite direktorije \shell{Rita}, \shell{Lucy}, \shell{Elenor}, \shell{Desmond}
		\item Kopirajte direktorij \shell{Rita} u \shell{Lucy}
		\item U direktoriju \shell{Desmond} stvorite prazne datoteke \shell{Molly} i \shell{Bill}
		\item U direktoriju \shell{Elenor} napravite simboličku poveznicu na direktorij \shell{Lucy} 
		\item Izbrišite direktorij \shell{Lucy}
		\item Provjerite sadržaj direktorija \shell{Elenor}
		\item Ispišite zadnje vrijeme promjene direktorija \shell{Elenor}
	\end{itemize}
\end{itemize}
\end{frame}

\begin{frame}[t]
\frametitle{Zadatak 4}
\begin{itemize}
	\item Četvrti zadatak:
	\begin{itemize}
		\item Očitajte vrijeme zadnjeg pristupa datoteci \shell{/etc/passwd}
		\item Napravite direktorij \shell{Pennie}
		\item Ispišite zauzeće direktorija
		\item U direktoriju \shell{Pennie} jednom naredbom napravite datoteke \shell{Sally} i \shell{Jules}
		\item U direktoriju \shell{Pennie} napravite simboličku poveznicu koja pokazuje na direktorij \shell{/usr/bin}
		\item Ponovno ispišite zauzeće direktorija
		\item Ispišite zauzeće direktorija uz dereferenciranje poveznica
		\item Izbrišite sve stvorene datoteke
	\end{itemize}
\end{itemize}
\end{frame}

\end{document}
