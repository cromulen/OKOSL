\documentclass[table,usenames,dvipsnames]{beamer}

\usepackage[english]{babel}
\usepackage[utf8]{inputenc}
\usepackage{listings}
\usepackage{datetime}
\usepackage{graphics}
\usepackage{fancybox}
\usepackage{color}
\usepackage[normalem]{ulem}
\usepackage{tikz}
\usepackage{verbatim}
\usepackage{varwidth}
\usetikzlibrary{shapes,arrows}
\usetheme{CambridgeUS}
\usecolortheme{seagull}
% Changing of bullet foreground color not possible if {itemize item}[ball]
\DefineNamedColor{named}{Purple}{cmyk}{0.52,0.97,0,0.55}
\setbeamertemplate{itemize item}[triangle]
\setbeamercolor{title}{fg=Purple}
\setbeamercolor{frametitle}{fg=Purple}
\setbeamercolor{itemize item}{fg=Purple}
\setbeamercolor{section number projected}{bg=Purple,fg=white}
\setbeamercolor{subsection number projected}{bg=Purple}

\renewcommand{\dateseparator}{.}
\newcommand{\todayiso}{\twodigit\day \dateseparator \twodigit\month \dateseparator \the\year}
\newcommand{\shell}[1]{\texttt{#1}}
\newcommand{\MyCBox}[1]{%
  \fcolorbox{black}{gray!30}{\begin{varwidth}{\dimexpr\linewidth-2\fboxsep}#1
\end{varwidth}}}
\definecolor{LightGray}{gray}{0.9}

\title{Osnove korištenja operacijskog sustava Linux}
\subtitle{08. Zamjenski znakovi i regularni izrazi} 
\author[Dino Lukman i Goran Cetušić]{Dino Lukman i Goran Cetušić\\{\small Nositelj: dr. sc. Stjepan Groš}}
\institute[FER]{Sveučilište u Zagrebu \\
				Fakultet elektrotehnike i računarstva}
				
\date{\todayiso}

\begin{document}
    %\beamerdefaultoverlayspecification{<+->}
{
\setbeamertemplate{headline}[] % still there but empty
\setbeamertemplate{footline}{}

\begin{frame}
\maketitle
\end{frame}
}

\begin{frame}
\frametitle{Sadržaj}
\tableofcontents
\end{frame}

\section{Zamjenski znakovi}
\begin{frame}[t]
\frametitle{Zamjenski znakovi (1)}
\begin{itemize}
  \item engl. wildcards
  \item Koriste se za brzo i efikasno pretraživanje i izvršavanje naredbi
  \begin{itemize}
    \item Osnovni zamjenski znakovi
    \begin{tabular}{c l}
      \shell{?}   & Predstavlja točno jedan znak \\
      \shell{*}   & Predstavlja nula ili više znakova       \\
      \shell{[]}  & Grupa koja se tretira kao jedan znak
    \end{tabular}
  \end{itemize}  
\end{itemize}
\end{frame}

\begin{frame}[t]
\frametitle{Zamjenski znakovi (2)}
\begin{itemize}
  \item Kako biste izlistali sve datoteke koje započinju s b i nalaze se u
        \shell{/bin} direktoriju?
  \item Jedna mogućnost je izlistati sve, a potom prepisivati jedan po 
        jedan
  \item Možemo koristiti zamjenske znakove  
\end{itemize}
\end{frame}

\begin{frame}[t]
\frametitle{Korištenje zamjenskih znakova (1)}
\begin{itemize}
  \item Kako bi izlistali sve datoteke koje započinju s b i nalaze se u 
        \shell{/bin} direktoriju?
  \begin{itemize}
    \item[] \shell{\$ ls -l /bin/b*}
  \end{itemize}
  \item Izlistati sve naredbe u \shell{/bin} direktoriju koje se sastoje od
        dva znaka
  \begin{itemize}
    \item[] \shell{\$ ls -l /bin/??}
  \end{itemize}
  \item Izlistati sve datoteke u \shell{/bin} direktoriju koje završavaju 
        slovom d
  \begin{itemize}
    \item[] \shell{\$ ls -l /bin/*d}
  \end{itemize}
\end{itemize}
\end{frame}

\begin{frame}[t]
\frametitle{Korištenje zamjenskih znakova (2)}
\begin{itemize}
  \item Ispisati sve datoteke u \shell{/bin} direktoriju koje započinju s
        a, b ili c
  \begin{itemize}
    \item Jedna varijanta
    \item[] \shell{\$ ls -l /bin/a* /bin/b* /bin/c*}
    \item Kraća i efikasnija varijanta
    \item[] \shell{\$ ls -l /bin/[abc]*}
  \end{itemize}
  \item Izlistati sve datoteke u \shell{/bin} direktoriju koje u sebi 
        sadrže barem jednu znamenku
  \begin{itemize}
    \item[] \shell{\$ ls -l /bin/*[0123456789]*}
  \end{itemize}
\end{itemize}
\end{frame}

\begin{frame}[t]
\frametitle{Korištenje zamjenskih znakova (3)}
\begin{itemize}
  \item Izlistati sve datoteke u \shell{/bin} direktoriju koje u sebi 
        sadrže barem jednu znamenku
  \begin{itemize}
    \item Efikasnija varijanta
    \item[] \shell{\$ ls -l /bin/*[0-9]*}
  \end{itemize}
  \item Moguće je zadavati raspone ASCII znakova!
  \item Zadatak
  \begin{itemize}
    \item Izlistati sve datoteke u \shell{/usr/bin} direktoriju koje 
          započinju sa svim slovima abecede osim s a
  \end{itemize}
\end{itemize}
\end{frame}

\begin{frame}[t]
\frametitle{Korištenje zamjenskih znakova (4)}
\begin{itemize}
  \item Moguće je upotrebom znaka \textasciicircum{} invertirati skup 
        znakova u zagradi 
  \item Izlistati sve datoteke u direktoriju \shell{/usr/bin} koje ne 
        započinju sa malim slovom abecede
  \begin{itemize}
    \item \shell{\$ ls -l /usr/bin/[\textasciicircum{}a-z]*}
  \end{itemize}
  \item Ako treba znakove \textasciicircum{}, \shell{-} i \shell{$]$} 
        tretirati kao ``obične'' znakove tada \textasciicircum{} ne smije 
        biti prvi, tj. odmah nakon otvorene uglate zagrade dok \shell{-} 
        mora biti prvi ili zadnji znak
\end{itemize}
\end{frame}

\begin{frame}[t]
\frametitle{Korištenje zamjenskih znakova (5)}
\begin{itemize}
  \item Zamjenski znakovi mogu se koristiti kod svih naredbi koje 
        prihvaćaju datoteke ili direktorije kao argument!
  \begin{itemize}
    \item Ljuska \textbf{prije} pokretanja naredbi uklanja zamjenske 
          znakove!
    \item Pronalazi sve datoteke koje odgovaraju izrazu i stavlja ih 
          umjesto zamjenskog izraza kao da su direktno uneseni
    \item Potom pokreće naredbu koja ne zna ništa o zamjenskim znakovima!
  \end{itemize}
\end{itemize}
\end{frame}

\begin{frame}[t]
\frametitle{Korištenje zamjenskih znakova (6)}
\begin{itemize}
  \item Zadaci
  \begin{itemize}
    \item Izlistaj datoteke u \shell{/bin/} koje završavaju nizom ``ep''
    \item Izlistaj detaljan izvještaj o svim jednoznamenkastim 
          direktorijima (ne njihov sadržaj) u \shell{/proc}
    \item Sortiraj po numeričkom iznosu veličine svih datoteka u 
          \shell{/usr/share/man/man1} koje počinju sa nizom “perl”
  \end{itemize}
\end{itemize}
\end{frame}

\begin{frame}[t]
\frametitle{Korištenje zamjenskih znakova (7)}
\begin{itemize}
  \item Rješenja
  \begin{itemize}
    \item \shell{ls /bin/*ep}
    \item \shell{ls -dl /proc/[0-9]}
    \item \shell{ls -l /usr/share/man/man1/perl* | awk '{print \$5}' | sort -n}
  \end{itemize}
\end{itemize}
\end{frame}

\section{Isključenje značenja posebnih znakova}
\begin{frame}[t]
\frametitle{Isključenje značenja posebnih znakova}
\begin{itemize}
  \item Ponekad ne želimo posebno značenje zamjenskih znakova
  \begin{itemize}
    \item Što ako baš imamo datoteku koja se zove \shell{*?}
    \item U tom slučaju upotrebljavamo navodnike ili znak 
          \shell{\char`\\}
  \end{itemize}
  \item Primjeri
  \begin{itemize}
    \item[] \shell{\$ ls -l /bin/b*}
    \item[] \shell{\$ ls -l "/bin/b*"}
    \item[] \shell{\$ ls -l /bin/b\char`\\{}*}
  \end{itemize}
\end{itemize}
\end{frame}

\section{Regularni izrazi}
\begin{frame}[t]
\frametitle{Regularni izrazi (1)}
\begin{itemize}
  \item Primjer
  \begin{itemize}
    \item Izlistati sve datoteke u \shell{/usr/bin} direktoriju koje 
          započinju s ab, bi ili ci
    \item Teško ili gotovo nemoguće sa zamjenskim znakovima!
  \end{itemize}
  \item Regularni izrazi
  \begin{itemize}
    \item Korištenje znakova i operatora te pravila regularnih izraza za 
          obradu teksta, pretragu, leksičku analizu, \ldots
    \item Moćno i kompleksno proširenje zamjenskih znakova
  \end{itemize}
\end{itemize}
\end{frame}

\begin{frame}[t]
\frametitle{Regularni izrazi (2)}
\begin{table}[h]
\rowcolors{1}{White}{LightGray}
\begin{tabular}{c l}
  \shell{c} & Znak c \\
  \textbackslash{}c & Čita znak kao slovo c, a ne kao specijalan znak \\
  \textasciicircum{} & Početak reda \\
  \shell{\$} & Kraj reda \\
  \shell{.} & Bilo koji znak \\
  \shell{[xy]} & Bilo koji znak u setu \\
  \shell{[\textasciicircum{}xy]} &  Bilo koji znak koji nije u setu \\
  \shell{c*} &Nijedno ili više pojavljivanja izraza iza kojeg se nalazi \\
  \shell{c+} & Jedno ili više pojavljivanja izraza iza kojeg se nalazi \\
  \shell{?} & Nijedno ili jedno pojavljivanje izraza iza kojeg se nalazi \\
  \shell{|} & Operator izbora
\end{tabular}
\end{table}
\end{frame}
  
\begin{frame}[t]
\frametitle{Primjeri regularnih izraza (1)}
\begin{itemize}
  \item Regularan izraz koji će zamjenjivati niz znakova ako koji se nalazi na
        početku retka datoteke
  \begin{itemize}
    \item[] \textasciicircum{}ako
  \end{itemize}
  \item Kreirati regularan izraz koji će zamjenjivati niz znakova \emph{sigh}
        koji se nalazi na kraju retka datoteke
  \begin{itemize}
    \item[] sigh\$
  \end{itemize}
\end{itemize}
\end{frame}

\begin{frame}[t]
\frametitle{Primjeri regularnih izraza (2)}
\begin{itemize}
  \item Linija koja ne završava sa slovom A 
  \begin{itemize}
    \item[] .*[\textasciicircum{}aA]
  \end{itemize}
  \item Cijeli broj u C-u:
  \begin{itemize}
    \item[] [0-9]+
  \end{itemize}
  \item Varijabla u C-u:
  \begin{itemize}
    \item[] [a-zA-Z]([a-zA-Z0-9])*
  \end{itemize}
\end{itemize}
\end{frame}

\begin{frame}[t]
\frametitle{Korištenje regularnih izraza}
\begin{itemize}
  \item Regularne izraze prihvaća mnoštvo programa
  \begin{itemize}
    \item Postoji i prošireni regularni izrazi!
  \end{itemize}
  \item Mi ćemo se pozabaviti naredbom \shell{grep}
  \item Postoji nekoliko varijanti te naredbe
  \begin{tabular}{l l}
    grep  &  Koristi regularne izraze \\
    egrep &  Koristi proširene regularne izraze
  \end{tabular}
\end{itemize}
\end{frame}
  
\begin{frame}[t]
\frametitle{Primjeri korištenja regularnih izraza (1)}
\begin{columns}[c]
\column{0.7\textwidth}
\begin{itemize}
  \item Pretpostavite kako je niz znakova s desne strane upisan u datoteku 
        \shell{regular}
  \item Odredite koji izlaz će dati sljedeći niz naredbi
  \begin{itemize}
    \item[] \shell{egrep a.e regular}
    \item[] \shell{egrep a.+e regular}
    \item[] \shell{egrep ab*e regular}
    \item[] \shell{egrep "(ab|cd)e" regular}
    \item[] \shell{egrep "(a|c).+e\$" regular}
  \end{itemize}
\end{itemize} 
\column{0.3\textwidth}
  \centering
  \MyCBox{abe \\ abbe \\ cde \\ 45a678 \\ ae \\ cababb \\ 12345} 
\end{columns}
\end{frame}

\begin{frame}[t]
\frametitle{Primjeri korištenja regularnih izraza (2)}
\begin{itemize}
  \item Zadatak
  \begin{itemize}
    \item Kreirajte datoteku \shell{regular} i isprobajte prethodne izraze
    \item Prvi niz stavljamo u datoteku i istovremeno ju kreiramo
    \begin{itemize}
      \item[] \shell{\$ echo niz1 > regular}
    \end{itemize}
    \item Potom dodajemo proizvoljan broj nizova na sljedeći način
    \begin{itemize}
      \item[] \shell{\$ echo niz2 >> regular}
      \item[] \shell{\$ echo niz3 >> regular}
    \end{itemize}
  \end{itemize}
\end{itemize}
\end{frame}

\begin{frame}[t]
\frametitle{Klase znakova (1)}
\begin{itemize}
  \item Primjer
  \begin{itemize}
    \item Izlistajte linije datoteke koje sadrže samo brojeve
    \item Prvi način
    \begin{itemize}
      \item[] \shell{\$ grep [0-9]* datoteka}
    \end{itemize}
    \item Drugi način
    \begin{itemize}
      \item[] \shell{\$ grep [[:digit:]]* datoteka}
    \end{itemize}
  \end{itemize}
\end{itemize}
\end{frame}

\begin{frame}[t]
\frametitle{Klase znakova (2)}
\begin{itemize}
  \item \shell{[:alnum:]} znakovi abecede ili brojevi. A-Za-z0-9
  \item \shell{[:alpha:]} znakovi abecede. A-Za-z 
  \item \shell{[:blank:]} praznina ili tabularni znak
  \item \shell{[:cntrl:]} kontrolni znakovi
  \item \shell{[:digit:]} brojevi. 0-9
  \item \shell{[:lower:]} mala slova abecede. a-z
  \item \shell{[:upper:]} velika slova abecede. A-Z
  \item \shell{[:space:]} praznina i tabularni znak
  \item \shell{[:print:]} znakovi koji mogu biti ispisani. Podrazumijeva 
        znakove u ASCII rasponu 32 – 126 uključujući prazan znak.
\end{itemize}
\end{frame}

\begin{frame}[t]
\frametitle{Zadaci (1)}
\begin{itemize}
  \item Ispisati retke u datoteci \shell{/etc/inittab} koji imaju u sebi 
        niz sačinjen od jednog slova i jednog ili više brojki
  \item Sljedeći zadaci se odnose na traženje uzoraka u datoteci 
        \shell{/usr/share/dict/words}
  \begin{itemize}
    \item Pronaći retke koji počinju sa ``ok''
    \item Koliko ima riječi u kojima se dva puta ponavlja niz ``ss''
  \end{itemize}
\end{itemize}
\end{frame}

\begin{frame}[t]
\frametitle{Zadaci (2)}
\begin{itemize}
  \item Koliko ima riječi od 3 slova koje počinju slovima a, b, c; 
        završavaju sa x, y, z?
  \item Koliko ima riječi koje počinju slovom ``a'', završavaju slovom 
        ``y'', a bilo gdje između sadrže niz ``bu''?
\end{itemize}
\end{frame}

\begin{frame}[t]
\frametitle{Rješenja (1)}
\begin{itemize}
  \item \shell{grep -E "\textasciicircum{}[0-9]+[a-Z][0-9]*\$|\textasciicircum{}[0-9]*[a-Z][0-9]+\$" /etc/inittab}
  \item \shell{grep -E \textasciicircum{}ok /usr/share/dict/words}
  \item \shell{grep -E .*ss.*ss.* /usr/share/dict/words | wc -l}
\end{itemize}
\end{frame}

\begin{frame}[t]
\frametitle{Rješenja (2)}
\begin{itemize}
  \item \shell{grep -E \textasciicircum{}[abc].[xyz]\$ /usr/share/dict/words | wc -l}
  \item \shell{grep -E \textasciicircum{}a.*bu.*y\$ /usr/share/dict/words | wc -l}
\end{itemize}
\end{frame}

\begin{frame}[t]
\frametitle{Literatura}
\begin{itemize}
  \item[] \small\url{http://www.rexegg.com/regex-quickstart.html}
  \item[] \small\url{http://www.regexr.com}
  \item[] \small\url{http://regexone.com}
  \item[] \small\url{http://regex.learncodethehardway.org/book/}
  \item[] \small\url{http://www.rexegg.com}
  \item[] \small\url{http://www.regular-expressions.info/conditional.html} 
  \item[] \small\url{http://tldp.org/LDP/Bash-Beginners-Guide/html/chap_04.html}
\end{itemize}
\end{frame}

\end{document}
