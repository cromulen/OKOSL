\documentclass{beamer}

\usepackage[english]{babel}
\usepackage[utf8]{inputenc}
\usepackage{listings}
\usepackage{datetime}
\usepackage{graphics}
\usepackage{fancybox}
\usepackage{color}
\usepackage[normalem]{ulem}
\usepackage{tikz}
\usetikzlibrary{shapes,arrows}
\usetheme{CambridgeUS}
\usecolortheme{seagull}
% Changing of bullet foreground color not possible if {itemize item}[ball]
\DefineNamedColor{named}{Purple}{cmyk}{0.52,0.97,0,0.55}
\setbeamertemplate{itemize item}[triangle]
\setbeamercolor{title}{fg=Purple}
\setbeamercolor{frametitle}{fg=Purple}
\setbeamercolor{itemize item}{fg=Purple}
\setbeamercolor{section number projected}{bg=Purple,fg=white}
\setbeamercolor{subsection number projected}{bg=Purple}

\renewcommand{\dateseparator}{.}
\newcommand{\todayiso}{\twodigit\day \dateseparator \twodigit\month \dateseparator \the\year}

\title{Osnove korištenja operacijskog sustava Linux}
\subtitle{00. Organizacija vještine}
\author[Dominik Barbarić]{Dominik Barbarić\\{\small Nositelj: dr. sc. Stjepan Groš}}
\institute[FER]{Sveučilište u Zagrebu \\
				Fakultet elektrotehnike i računarstva}

\date{\todayiso}

\begin{document}
    %\beamerdefaultoverlayspecification{<+->}
{
\setbeamertemplate{headline}[] % still there but empty
\setbeamertemplate{footline}{}

\begin{frame}
\maketitle
\end{frame}
}

\section{Cilj vještine}
\begin{frame}[t]
	\frametitle{Cilj vještine}
	\begin{itemize}
		\item Upoznati studente s radom u komandnolinijskom sučelju
		\item Dati pregled mogućnosti Linuxa
		\item Dobiti osjećaj za rješavanje problema koji nisu pokriveni vještinom
		\item Definirati znanja potrebna za NKOSL
	\end{itemize}
\end{frame}

\begin{frame}[t]
	\frametitle{Cilj vještine}
	\begin{itemize}
		\item Nije potrebno predznanje
		\item Vještina je izborna
		\begin{itemize}
			\item Vještine se upisuju u ISVU van strukture studija i \textbf{ECTS bodovi ne ulaze} u zbrojeve za dom, plaćanje ili prijelaz na iduću godinu
			\item Vještinu upisujete radi vlastitog interesa i želje za učenjem rada na Linuxu
			\item Očekuje se angažiranost studenata, prijedlozi, komentari, kritike, \ldots
		\end{itemize}
	\end{itemize}
\end{frame}

\section{Sadržaj vještine}
\begin{frame}[t]
	\frametitle{Sadržaj vještine}
	\begin{itemize}
		\item Korištenje CLI
		\item Skripte
		\item Rad s datotekama i direktorijima
		\item Pretraživanje, filtri, cjevovodi
		\item Zamjenski znakovi i regularni izrazi
		\item Procesi
		\item Ljuske
		\item Korisnici i grupe
		\item Vlasništvo i dozvole
		\item \ldots
	\end{itemize}
\end{frame}

\section{Predavanja}
\begin{frame}[t]
\frametitle{Predavanja}
\begin{itemize}
	\item Svake subote osim vikendima prije ispitnih tjedana
	\item 8 termina
	\item A111 - 2 grupe
	\begin{itemize}
		\item 10-12 sati
		\item 12-14 sati
	\end{itemize}
	\item D272 - 2 grupe
	\begin{itemize}
		\item 10-12 sati
		\item 12-14 sati
	\end{itemize}
	\item Dva školska sata predavanja s vježbama i praktičnim radom
	\item Predavanja i zadaci za vježbu se objavljuju na FERwebu
\end{itemize}
\end{frame}

\section{Bodovanje}
\begin{frame}[t]
\frametitle{Bodovanje}
\begin{itemize}
	\item Bodovi:
	\begin{itemize}
		\item 5 domaćih zadaća x 7 bodova = 35 bodova
		\item 3 laboratorijske vježbe x 15 bodova = 45 bodova
		\item Prisustvo na predavanjima = 20 bodova
	\end{itemize}
	\item Uvjeti za prolaz vještine:
	\begin{itemize}
		\item Sve zadaće uspješno predane
		\item Predani i odgovarani svi labosi
		\item 10 bodova na prisustvu (50\% predavanja)
		\item Sve skupa \textbf{\textgreater 50} bodova
	\end{itemize}
\end{itemize}
\end{frame}

\begin{frame}[t]
\frametitle{Domaće zadaće}
\begin{itemize}
	\item Zadaju se na FERwebu poslije svakog predavanja
	\item U pravilu predaja do srijede u 23:59
	\item Način predaje bit će naknadno objavljen
	\item Zadaci su slični zadacima za vježbu obrađenim na predavanju
	\item Za svaku zadaću dobiva se povratna informacija od ispravljača
\end{itemize}
\end{frame}

\begin{frame}[t]
\frametitle{Laboratorijske vježbe}
\begin{itemize}
	\item Zadaju se nekoliko tjedana prije roka za predaju
	\item Sadrže gradivo koje obuhvaća predavanja do roka predaje
	\item U zadacima se traži razumijevanje i samostalno snalaženje
	\item Labosi se predaju prema uputama i odgovaraju usmeno
\end{itemize}
\end{frame}

\section{Radna okolina}
\begin{frame}[t]
\frametitle{Radna okolina}
\begin{itemize}
	\item Predavanja se sastoje od teorije, primjera i zadataka za vježbu
	\item Studenti zadatke rješavaju samostalno koristeći
	\begin{itemize}
		\item vlastito računalo (poželjno)
		\item ako dobijemo dvorane, računala dostupna na fakultetu
	\end{itemize}
	\item Studenti moraju doći s pripremljenom Linux distribucijom, preporučeno Debian, na vlastitom računalu ili s pripremljenim virtualnim strojem kojeg možete pokrenuti na fakultetskom računalu.
	\item Možete koristiti i live distribucije
\end{itemize}
\end{frame}

\section{Literatura}
\begin{frame}[t]
\frametitle{Literatura}
\begin{itemize}
	\item Jedna od pogodnosti Linuxa je jaka zajednica
	\item Nakon svakog predavanja na prezentaciji će biti objavljena dodatna literatura
	\item Koristite svu literaturu koja vam je na raspolaganju!
	\begin{itemize}
		\item Knjige, članci, druga predavanja, stripovi, \ldots :)
		\item Internet (najbolji izvor)
		\item wiki, irc, forumi, google
	\end{itemize}
\end{itemize}
\end{frame}

\end{document}
