\documentclass[table,usenames,dvipsnames]{beamer}

\usepackage[english]{babel}
\usepackage[utf8]{inputenc}
\usepackage{listings}
\usepackage{datetime}
\usepackage{graphics}
\usepackage{fancybox}
\usepackage{color}
\usepackage{courier}
\usepackage[normalem]{ulem}
\usepackage{tikz}
\usepackage{verbatim}
\usepackage{multirow}
\usepackage{fancyvrb}
\usetikzlibrary{shapes,arrows,positioning}
\usetheme{CambridgeUS}
\usecolortheme{seagull}
% Changing of bullet foreground color not possible if {itemize item}[ball]
\DefineNamedColor{named}{Purple}{cmyk}{0.52,0.97,0,0.55}
\setbeamertemplate{itemize item}[triangle]
\setbeamercolor{title}{fg=Purple}
\setbeamercolor{frametitle}{fg=Purple}
\setbeamercolor{itemize item}{fg=Purple}
\setbeamercolor{section number projected}{bg=Purple,fg=white}
\setbeamercolor{subsection number projected}{bg=Purple}

\renewcommand{\dateseparator}{.}
\newcommand{\todayiso}{\twodigit\day \dateseparator \twodigit\month \dateseparator \the\year}
\newcommand{\shell}[1]{\texttt{#1}}
\definecolor{LightGray}{gray}{0.9}

\title{Osnove korištenja operacijskog sustava Linux}
\subtitle{06. Vlasništvo i dozvole}
\author[Dominik Barbarić]{Dominik Barbarić\\{\small Nositelj: doc. dr. sc. Stjepan Groš}}
\institute[FER]{Sveučilište u Zagrebu \\
				Fakultet elektrotehnike i računarstva}
				
\date{\todayiso}

\begin{document}
    %\beamerdefaultoverlayspecification{<+->}
{
\setbeamertemplate{headline}[] % still there but empty
\setbeamertemplate{footline}{}

\begin{frame}
\maketitle
\end{frame}
}

\begin{frame}
\frametitle{Sadržaj}
\tableofcontents
\end{frame}

\section{Dozvole}
\begin{frame}[t]
\frametitle{Dozvole (1)}
\begin{itemize}
	\item Naredba \shell{ls -l} ispisuje informacije o vlasnicima i dozvolama objekta
	{\small \item[] \shell{\$ ls -l datoteka.txt}
	\item[] \shell{-\textbf{rw-r--r-- 1 pero users} 0 Jan  4 23:19 datoteka.txt}}
	\vfill
    \item Objekt je vlasništvo korisnika i grupe
    \begin{itemize}
		\item Drugo polje označava vlasnika - korisnika \hfill (\shell{pero}) \hfill \,
		\item Treće polje označava vlasnika - grupu \hfill (\shell{users}) \hfill \,
    \end{itemize}
    \item Prvo polje u prvom bitu sadrži oznaku tipa datoteke, a ostalih 9 bitova se nazivaju \textbf{mode} objekta
  \end{itemize}
\end{frame}

\begin{frame}[t]
\frametitle{Dozvole (2)}
\begin{itemize}
  \item \textbf{mode} definira dozvoljene operacije na svakom objektu
  \vspace{1em}
  \item Devet bitova dijele se u tri grupe od koji svaka čini jedan troznamenkasti binarni broj
  \item Svaki troznamenkasti binarni broj se može prikazati jednom oktalnom znamenkom
\end{itemize}
\begin{itemize}
  \item Svaka oktalna znamenka modea predstavlja skup dozvola koje su dodijeljene sljedećim korisnicima objekta i to:
  \begin{itemize}
	\item Prva oktalna znamenka definira prava za vlasnika - korisnika \hfill \textbf{user}
    \item Druga oktalna znamenka definira prava za vlasnika - grupu \hfill \textbf{group}
    \item Treća oktalna znamenka definira prava za sve ostale \hfill \textbf{others}
  \end{itemize}
\end{itemize}
\end{frame}

\begin{frame}[t]
\frametitle{Dozvole (3)}
\begin{itemize}
  \item Značenja pojedinih bitova svake znamenke
    \begin{tabular}{l l}
     r & \textbf{read} - Dozvoljeno čitanje \\
     w & \textbf{write} - Dozvoljeno pisanje \\
     x & \textbf{execute} - Dozvoljeno izvršavanje / pretraživanje direktorija
    \end{tabular}
      \item Svaki pojedini bit može biti u stanju
      \begin{itemize}
      	\item \textbf{uključen} - operacija dozvoljena
      	\item \textbf{isključen} - operacija zabranjena
      \end{itemize}
\end{itemize}
\vfill
\begin{itemize}
  \item \textbf{Primjer 1}
  \begin{itemize}
    \item[] \shell{rwxr-xr-x} = $111101101_2$ = $755_8$
    \item[] \hspace{1em} \begin{tabular}{l c c c}
    	& \textbf{r} & \textbf{w} & \textbf{x}\\
    	\textbf{user} & + & + & +\\
    	\textbf{group} & + & - & +\\
    	\textbf{others} & + & - & +
    \end{tabular}
  \end{itemize}
\end{itemize}
\end{frame}

\begin{frame}[t]
	\frametitle{Dozvole (4)}
\begin{itemize}
	\item \textbf{Primjer 2}
	\begin{itemize}
	    \item[] \shell{rw-r--r--} = $644$
	    \item[] \hspace{1em} \begin{tabular}{l c c c}
	    	& \textbf{r} & \textbf{w} & \textbf{x}\\
	    	\textbf{user} & + & + & -\\
	    	\textbf{group} & + & - & -\\
	    	\textbf{others} & + & - & -
	    \end{tabular}
	\end{itemize}
\end{itemize}
\vfill
\begin{itemize}
	\item \textbf{Primjer 3}
	\begin{itemize}
	    \item[] \shell{r--r--r--} = $444$
	    \item[] \hspace{1em} \begin{tabular}{l c c c}
	    	& \textbf{r} & \textbf{w} & \textbf{x}\\
	    	\textbf{user} & + & - & -\\
	    	\textbf{group} & + & - & -\\
	    	\textbf{others} & + & - & -
	    \end{tabular}
	\end{itemize}
\end{itemize}
\end{frame}

\begin{frame}[t]
\frametitle{Promjena dozvola (1)}
\begin{itemize}
  \item Promjena modea obavlja se naredbom \shell{chmod}
  \begin{itemize}
    \item[] \shell{chmod <mode> <objekt>}
  \end{itemize}
  \item Mode se može zadati oktalno i simbolički
  \item Moguće jer rekurzivno mijenjati prava
  \begin{itemize}
    \item[] \shell{chmod -R <mode> <objekt>}
  \end{itemize}
  \vfill
   	\item Vlasnik datoteke može bez obzira na trenutni mod
   	\begin{itemize}
		\item promijeniti mode
   		\item obrisati datoteku
   	\end{itemize}
\end{itemize}
\end{frame}

\begin{frame}[t]
\frametitle{Promjena dozvola (2)}
\begin{itemize}
  \item \textbf{Primjer 4}
  \begin{itemize}
    \item[] \shell{chmod ugo=rwx file1}
    \item[] \hspace{1em} \begin{tabular}{l c c c}
    	& \textbf{r} & \textbf{w} & \textbf{x}\\
    	\textbf{user} & + & + & +\\
    	\textbf{group} & + & + & +\\
    	\textbf{others} & + & + & +
    \end{tabular}
    \item \vspace{1em} Alternativno:
    \item[] \shell{chmod a=rwx file1}
    \item[] \shell{chmod 777 file1}
  \end{itemize}
\end{itemize}
\begin{itemize}
	\item \textbf{Primjer 5}
    \begin{itemize}
	    \item[] \shell{chmod u=rwx,go=rx file1 file2}
	    \item[ili] \shell{chmod 755 file1 file2}
	\end{itemize}
\end{itemize}
\end{frame}


\begin{frame}[t]
	\frametitle{Promjena dozvola (3)}
\begin{itemize}
	\item \textbf{Primjer 6}
	\begin{itemize}
	    \item[] \shell{chmod g+w file1 file2 file3}
	    \item[] \hspace{1em} \begin{tabular}{l c c c}
    	& \textbf{r} & \textbf{w} & \textbf{x}\\
    	\textbf{user} & $\ast$ & $\ast$ & $\ast$\\
    	\textbf{group} & $\ast$ & + & $\ast$\\
    	\textbf{others} & $\ast$ & $\ast$ & $\ast$
		\end{tabular}
    \end{itemize}
	\item \textbf{Primjer 7}
	\begin{itemize}
	    \item[] \shell{chmod -x file1} 
	    \item[ili] \shell{chmod a-x file1}
	    \item[] \hspace{1em} \begin{tabular}{l c c c}
	        	& \textbf{r} & \textbf{w} & \textbf{x}\\
	        	\textbf{user} & $\ast$ & $\ast$ & -\\
	        	\textbf{group} & $\ast$ & $\ast$ & -\\
	        	\textbf{others} & $\ast$ & $\ast$ & -
	        \end{tabular}
    \end{itemize}
\end{itemize}
\end{frame}


\defverbatim[colored]\makeset{
	\begin{lstlisting}[language=bash, basicstyle=\footnotesize\ttfamily, showstringspaces=false, keywordstyle=\color{blue}]
#!/bin/bash
echo "Skripta je pokrenuta"
	\end{lstlisting}
}

\begin{frame}[fragile]
\frametitle{Izvršavanje datoteka}
\begin{itemize}
	\item Svaka datoteka na UNIX sustavu može biti izvršna (\textit{executable})
	\item Skripta se, tako, može izvršiti korištenjem zadanog interpretora
\end{itemize}
\begin{itemize}
	\item Postavljanjem \shell{x} dozvole svaka se datoteka može izvršiti izravnim pozivanjem
\end{itemize}
\vfill
\begin{block}{/home/linux/skripta.sh \hfill mode 755}
	\makeset
\end{block}
\vfill
\begin{Verbatim}[fontsize=\footnotesize]
~$ /home/linux/skripta.sh
Skripta je pokrenuta
~$ ./skripta.sh
Skripta je pokrenuta
\end{Verbatim}
\end{frame}


\begin{frame}
	\frametitle{Promjena dozvola (4)}
\begin{itemize}
	\item Naredba \shell{chmod} može prihvatiti poseban argument prilikom simboličkog zadavanja modea
\end{itemize}
\begin{itemize}
	\item[] \shell{X} (Capital X)
	 \begin{itemize}
	 	\item Direktorijima postavlja \shell{x} dozvolu
	 	\item Ostalim datotekama ne mijenja mod
	 	\item Omogućuje listanje direktorija bez dodavanja dozvole za izvršavanje datoteka
	 	\item Koristan prilikom rekurzivne promjene modea:
	 	\item[] \hspace{1em} \shell{chmod -R a+X dir1}
	 \end{itemize}
\end{itemize}
\end{frame}

\section{Posebne dozvole}
\begin{frame}[t]
	\frametitle{Posebne dozvole (1)}
	\framesubtitle{Sticky bit}
\textbf{Sticky bit} / Text mode
\begin{itemize}
	\item \textbf{Kod direktorija}
	\item[] Dozvoljava brisanje direktorija \textbf{samo} vlasniku i root korisniku
	\item \textbf{Kod datoteka}
	\item[] Nakon izvršavanja datoteke proces ostaje u memoriji
\end{itemize}
\begin{itemize}
	\item Simbolički se označava s \shell{t} na mjestu \shell{x} dozvole za \textit{others} korisnike
	\item[] {\footnotesize \shell{-rwxr--r-t 1 pero users 0 Jan  4 23:21 datoteka.txt} }
	\item Ako \textit{others} ujedno ima i \shell{x} dozvolu tada se sticky bit označava s velikim \shell{T}
\end{itemize}
\end{frame}

\begin{frame}[t]
	\frametitle{Posebne dozvole (2)}
	\framesubtitle{SUID i SGID}
\begin{itemize}
	\item Za razumijeti preostala dva posebna bita potrebno je shvatiti što se događa s dozvolama korisnika koji pokreće izvršnu datoteku
	\item Svaki proces se pokreće s UID i GID primarne grupe korisnika koji ga je pozvao. Pokrenuti proces ima sve ovlasti tog korisnika
\end{itemize}
\vfill
\textbf{Set user ID (SUID)} i \textbf{Set group ID (SGID)}
\begin{itemize}
	\item Postavljanjem ovih bitova u mode datoteke proces koji pokreće datoteku dobiva dozvole \textbf{vlasnika - korisnika} (SUID bit), odnosno \textbf{vlasnika - grupe} (SGID) izvršne datoteke
\end{itemize}
\end{frame}

\begin{frame}[t]
	\frametitle{Posebne dozvole (3)}
	\framesubtitle{SUID i SGID}
\begin{itemize}
	\item Simbolički se označava sa \shell{s} na mjestu \shell{x} dozvole za određenu grupu korisnika
	\item[] {\footnotesize \shell{-rwsr--r-x 1 pero users 0 Jan  4 23:21 datoteka.txt} \hfill SUID }
	\item[] {\footnotesize \shell{-rw-r-sr-x 1 pero users 0 Jan  4 23:21 datoteka.txt} \hfill SGID }
\end{itemize}
\begin{itemize}
	\item Primijetite da SUID, odnosno SGID ne impliciraju \shell{x} dozvolu vlasnicima datoteke. U gornjem primjeru samo \textit{others} imaju pravo izvršiti datoteku i u tom trenutku će isti dobiti prava vlasnika.
	\item Ako vlasnik, \textit{user} ili \textit{group} ujedno ima i \shell{x} dozvolu tada se posebni bitovi označavaju s velikim \shell{S}
\end{itemize}
\end{frame}

\section{Zadani mode}
\begin{frame}[t]
\frametitle{Zadani mode (1)}
\begin{itemize}
	\item Kreiranjem novog objekta on poprima zadani mode
	\item Definira ga trenutni filesystem i procesi koji kreiraju objekt
\end{itemize}
\begin{itemize}
	\item Primjenom \textbf{umask} mogu se ograničiti dozvole koje postavljaju nadređeni procesi
	\item umask ima isti format kao i mode, no s različitim značenjem bitova
	\begin{itemize}
		\item \textbf{1} - Isključuje dozvolu na poziciji bita
		\item \textbf{0} - Ne mijenja dozvolu koju je postavio nadležni proces
	\end{itemize}
\end{itemize}
\end{frame}

\begin{frame}[t]
	\frametitle{Zadani mode (2)}
\begin{itemize}
	\item Naredbom \shell{umask} se mijenja trenutni umask
	\begin{itemize}
		\item \textbf{Bez argumenata} - ispisuje trenutnu vrijednost u oktalnom obliku
		\item \textbf{Argument -S} - ispisuje trenutnu vrijednost u simboličkom obliku
		\item \textbf{Argument 4 oktalne znamenke} - mijenja vrijednost umaska
		\item[] \hspace{1em} Prva oktalna znamenka je za specijalne modove
	\end{itemize}
\end{itemize}
\begin{itemize}
	\item U datoteci s popisom montiranih datotečnih sustava, \shell{/etc/fstab} se mogu navesti tri vrste maski
	\begin{itemize}
		\item \textbf{umask} - Odnosi se na sve vrste datoteka
		\item \textbf{fmask} - Odnosi se na sve regularne datoteke
		\item \textbf{dmask} - Odnosi se na sve direktorije
	\end{itemize}
	\item Ove vrste maski se mogu navesti i prilikom korištenja naredbe \shell{mount}
\end{itemize}
\end{frame}
  
\section{Promjena vlasnika}
\begin{frame}[t]
\frametitle{Promjena vlasnika (1)}
\begin{itemize}
  \item Promjena vlasnika objekta obavlja se naredbom \shell{chown}
  \begin{itemize}
    \item[] \shell{chown <korisnik> <objekt>}
  \end{itemize}
	\item Promjena grupe objekta obavlja se naredbom \shell{chgrp}
	\begin{itemize}
		\item[] \shell{chgrp <grupa> <objekt>}
		\item[ili] \shell{chown :<grupa> <objekt>}
	\end{itemize}
\end{itemize}
\vfill
\begin{itemize}
  \item Moguće je istovremeno promijeniti korisnika i grupu
  \begin{itemize}
    \item[] \shell{\$ chown <korisnik>:<grupa> <objekt>}
    \item[] \vspace{1em} \shell{\$ chown <korisnik>: <objekt>}
	\item[] Postavlja korisnika i grupu koja odgovara primarnoj grupi korisnika
  \end{itemize}
\end{itemize}
\end{frame}

\end{document}